\subsection{CPPSLOW-FUN-10-35: Synthetic broadband flat generation}
\label{cppslow-fun-10-35}

\subsubsection{Requirements}

DMS-REQ-0130.

\subsubsection{Test items}

This test will check:

\begin{itemize}

  \item{That a pipeline task (or equivalent tool) exists which generates a
  synthetic broad-band flat for each filter in use with the LSST system.}

\end{itemize}

\subsubsection{Intercase dependencies}

\hyperref[cppslow-fun-10-20]{CPPSLOW-FUN-10-20}.

\subsubsection{Input specification}

\begin{note}
Detailed specification of the inputs required will require further thought \&
input from the Calibration Scientist; this is a work in progress.
\end{note}

\begin{itemize}

  \item{Monochromatic flat field data cube covering all filters.}

\end{itemize}

These products should be available within a Butler repository accessible
through the regular LSST data access framework from the system on which the test
is being run.

\subsubsection{Output specification}

\begin{itemize}

  \item{One synthetic broadband flat field image for the CCD being tested for
  each filter.}

\end{itemize}

These products should be persisted to a Butler repository accessible through
the regular LSST data access framework from the system on which the test is
being run.

\subsubsection{Procedure}

The task (or tasks, if one per filter; final design TBD) for generating the
broadband synthetic flat field image will be executed from the command
line, with a configuration appropriate for it to fetch required input data
from the input Butler repository.

The resulting broadband synthetic flats will be persisted to the output
repository. They should be retreived from the output repository using the
Butler and checked to ensure they contain physically plausible values (TBD by
the Calibration Scientist).
