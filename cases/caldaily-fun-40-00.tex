\subsection{CALDAILY-FUN-40-00: Bad pixel map generation}
\label{caldaily-fun-40-00}

\subsubsection{Requirements}

DMS-REQ-0101.

\subsubsection{Test items}

This test will check:

\begin{itemize}

  \item{That the daily calibration products update produces a report
  describing the evolution of calibration products from night to night;}

  \item{The daily calibration products report must include a broadband flat in
  each filter.}

\end{itemize}

\subsubsection{Intercase dependencies}

\begin{itemize}

  \item{\hyperref[cppslow-fun-10-35]{CPPSLOW-FUN-10-35}}

\end{itemize}

\subsubsection{Input specification}

As for \hyperref[cppslow-int-20-00]{CPPSLOW-INT-20-00}, but corresponding to
\emph{two} nights of data acquisition.

These products should be available within a Butler repository accessible
through the regular LSST data access framework from the system on which the test
is being run.

\subsubsection{Output specification}

\begin{itemize}

  \item{A calibration report describing the evolution of the calibration
  products between the two nights under test.}

\end{itemize}

\subsubsection{Procedure}

The daily calibration products update pipeline is invoked for the first night
under test, and the results persisted to an output repository.

The daily calibration products update pipeline is invoked for the second night
under test, and the results persisted to an output repository.

The operator should be presented (in a form TBD) with a test report,
describing the evolution of the calibration products from the first to the
second night. The detailed contents of this report are TBD.

The operator should verify that broadband flats, in each filter, are available
from the output repository for both nights under test.
