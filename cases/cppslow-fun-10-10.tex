\subsection{CPPSLOW-FUN-10-10: Crosstalk correction matrix generation}
\label{cppslow-fun-10-10}

\subsubsection{Requirements}

DMS-REQ-0061,DMS-REQ-0130.

\subsubsection{Test items}

This test will check:

\begin{itemize}

  \item{That a pipeline task (or equivalent tool) exists which generates a
  matrix describing the fraction of the signal detected in any given amplifier
  on each sensor in the focal plane appears in any other amplifier.}

\end{itemize}

Note that crosstalk is sensitive to the details of the camera configuration
(circuit board locations, cable flex, etc), and so the final values of the
crosstalk correction matrix cannot be measured until the camera is in situ on
the mountain (and even then they may continue to evolve, necessitating periodic
re-measurement). However, this test verifies the operation of the algorithm for
generating the matrix, not the values used in operation, so this test does not
need to be run with the camera in its final configuration.

\subsubsection{Intercase dependencies}

None.

\subsubsection{Input specification}

\begin{note}
Detailed specification of the inputs required will require further thought \&
input from the Calibration Scientist; this is a work in progress.
\end{note}

\begin{itemize}

  \item{Dithered Colliated Beam Projector (CBP) observations with the full
  camera or a representative subset thereof.}

\end{itemize}

These products should be available within a Butler repository accessible
through the regular LSST data access framework from the system on which the test
is being run.

\subsubsection{Output specification}

\begin{itemize}

  \item{A crosstalk correction matrix.}

\end{itemize}

These products should be persisted to a Butler repository accessible through
the regular LSST data access framework from the system on which the test is
being run.

\subsubsection{Procedure}

The task for generating the crosstalk correction matrix will be executed from the
command line, with a configuration appropriate for it to fetch required input
data from the input Butler repository.

The resulting crosstalk correction matrix will be persisted to the output
repository. It should be retrieved from the output repository using the Butler
and checked to ensure it contains physically plausible values (TBD by the
Calibration Scientist).
