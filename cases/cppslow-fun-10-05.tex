\subsection{\textsc{cppslow-fun-10-05}: Bias residual image generation}
\label{cppslow-fun-10-05}

\subsubsection{Requirements}

DMS-REQ-0060,DMS-REQ-0130.

\subsubsection{Test items}

This test will check:

\begin{itemize}

  \item{That a pipeline task (or equivalent tool) exists which generates a
  master image which can be used to correct for temporally stable bias structure
  in data from a CCD.}

\end{itemize}

\subsection{Intercase dependencies}

None.

\subsubsection{Input specification}

\begin{note}
Detailed specification of the inputs required will require further thought \&
input from the Calibration Scientist; this is a work in progress.
\end{note}

\begin{itemize}

  \item{Multiple (how many?) zero-second exposures of the CCD under test.}

\end{itemize}

These products should be available within a Butler repository accessible
through the regular LSST data access framework from the system on which the test
is being run.

\subsubsection{Output specification}

\begin{itemize}

  \item{A master bias residual image.}

\end{itemize}

These products should be persisted to a Butler repository accessible through
the regular LSST data access framework from the system on which the test is
being run.

\subsubsection{Procedure}

The task for generating the master bias residual image will be executed from the
command line, with a configuration appropriate for it to fetch required input
data from the input Butler repository.

The resulting master bias will be persisted to the output repository. It should
be retrieved from the output repository using the Butler and checked to ensure
it contains physically plausible values (TBD by the Calibration Scientist).
