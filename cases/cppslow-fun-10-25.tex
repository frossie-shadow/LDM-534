\subsection{\textsc{cppslow-fun-10-25}: Dark current correction frame generation}
\label{cppslow-fun-10-25}

\subsubsection{Requirements}

DMS-REQ-0063,DMS-REQ-0282.

\subsubsection{Test items}

This test will check:

\begin{itemize}

  \item{That a pipeline task (or equivalent tool) exists which generates a
  coadded dark current correction image.}

\end{itemize}

\subsubsection{Intercase dependencies}

\hyperref[cppslow-fun-10-05]{CPPSLOW-FUN-10-05}.

\subsubsection{Input specification}

\begin{note}
Detailed specification of the inputs required will require further thought \&
input from the Calibration Scientist; this is a work in progress.
\end{note}

\begin{itemize}

  \item{Multiple individual dark exposures of a single CCD with exposure times
  of 300\,s.}

\end{itemize}

These products should be available within a Butler repository accessible
through the regular LSST data access framework from the system on which the test
is being run.

\subsubsection{Output specification}

\begin{itemize}

  \item{A coadded dark current correction frame.}

\end{itemize}

These products should be persisted to a Butler repository accessible through
the regular LSST data access framework from the system on which the test is
being run.

\subsubsection{Procedure}

\begin{note}
LDM-151 notes that we need to perform standard ISR on the darks before
combining them, so that likely means a call out to a single frame test case.
\end{note}

The instrument signature removal code will be run on the individual input
exposures from the command line, and the results persisted to a data
repository.

The task for generating the coadded dark current correction frame will be
executed from the command line, and the results persisted to a further data
repository.

The Butler will be used to retrieve the flat field data cube from the output
repository, and the contents checked to ensure they are physically plausible
(values TBD by the Calibration Scientist.)
