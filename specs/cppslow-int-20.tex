\subsection{CPPSLOW-INT-20: Periodic Calibration Product Production Integration}
\label{cppslow-int-20}

\subsubsection{Objective}

This test design verifies that all the constituent algorithms of the periodic
CPP payload, tested separately in \hyperref[cppslow-fun-10]{CPPSLOW-FUN-10}
(\S\ref{cppslow-fun-10}), can be integrated and controlled by the LSST
processing control system.

\subsubsection{Approach refinements}

The general approach defined in \citeds{LDM-503} is used.

\subsubsection{Test case identification}

\begin{longtable} {|p{0.4\textwidth}|p{0.6\textwidth}|}\hline
\textbf{Test Case}  & \textbf{Description} \\\hline

\hyperref[cppslow-int-20-00]{CPPSLOW-INT-20-00} & Tests that an end-to-end calibration products pipeline can be run under manual control.\\\hline
\hyperref[cppslow-int-20-05]{CPPSLOW-INT-20-05} & Tests that the end-to-end calibration products pipeline can process data from the camera guider sensors.\\\hline
\hyperref[cppslow-int-20-10]{CPPSLOW-INT-20-10} & Tests that a complete periodic calibration products productionc an be run under the control of the LSST system.\\\hline
\end{longtable}
