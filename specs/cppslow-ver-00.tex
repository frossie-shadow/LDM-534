\subsection{CPPSLOW-VER-00: Calibration Product Verification}
\label{cppslow-ver-00}

\subsubsection{Objective}

This test design verifies that the calibration products production pipeline as
designed and built meets the overall requirements of the DM system.
Specifically, we verify that:

\begin{itemize}

  \item{The design of the system is such that all calibration products
  required by \citeds{LSE-61} are produced;}

  \item{The code as delivered is accompanied by a suite of unit tests;}

  \item{The code as delivered is accompanied by appropriate documentation;}

  \item{The code complies with all relevant DM coding
  standards\footnote{\url{https://developer.lsst.io/coding/intro.html}};}

  \item{The code makes use of standard DM interfaces to e.g. the data
  backbone, the logging system, the provenance system;}

  \item{The code is built and tested by the DM continuous integration system.}

\end{itemize}

Note that the tests described in this section apply to all periodically
executed calibration products production payloads, regardless of cadence (the
same codebase will be used for daily updates and annual calibration products
production).

\subsubsection{Approach refinements}

The general approach defined in \citeds{LDM-503} is used. Methods include:

\begin{itemize}

  \item{Document inspection;}
  \item{Code inspection;}
  \item{Review of CI system logs.}

\end{itemize}

\subsubsection{Test case identification}
\phantom{ } % This is necessary to suppress an ugly page break for reasons I've not understood.
\begin{longtable} {|p{0.4\textwidth}|p{0.6\textwidth}|}\hline
\textbf{Test Case}  & \textbf{Description} \\\hline
\hyperref[cppslow-ver-00-00]{CPPSLOW-VER-00-00} & CPP design inspection \\\hline
\hyperref[cppslow-ver-00-05]{CPPSLOW-VER-00-05} & CPP code inspection \\\hline
\hyperref[cppslow-ver-00-10]{CPPSLOW-VER-00-10} & CPP testing review \\\hline
\end{longtable}
